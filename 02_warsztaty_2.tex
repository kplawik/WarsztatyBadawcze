\section{Tematyka}

Zastosowanie sztucznej inteligencji do przewidywania struktury molekularnej antybiotyków wobec konkretnych, wybranych bakterii zaklasyfikowanych przez WHO w 2024 roku jako te najbardziej zagrażające ludzkiemu życiu.

\subsection{Argumenty}

\begin{enumerate}
    \item Tematyka związana z zastosowaniem sztucznej inteligencji ogólnie w medycynie staje się coraz częściej poruszana w artykułach naukowych ze względu na coraz lepsze algorytmy, umożliwiające uzyskanie bardzo dokładnych i precyzyjnych wyników.
    \item Niezwykle ważna tematyka z powodu wzrastającej odporności wielu bakterii na aktualnie stosowane antybiotyki, która przyczynia się rocznie do ponad 1 mln śmierci na świecie oraz może przyczynić się do 2050 roku do nawet 10 mln śmierci rocznie.
    \item Przewidywanie struktury molekularnej antybiotyków może przyczynić się do znacznego przyśpieszenia skomplikowanego i długotrwałego procesu ich odkrywania tradycyjnymi metodami, a w konsekwencji do zmniejszenia związanych z nim kosztów.
    \item Możliwość przewidywania struktur molekularnych antybiotyków znacznie odmiennych od tych dotychczas znanych pozwala uniknąć redundancji, a w konsekwencji zredukować czas i środki finansowe poświęcane na odkrywanie nowych antybiotyków.
\end{enumerate}

Na podstawie https://www.nature.com/articles/s41598-025-91190-x.

\subsection{Rozwinięcie tematyki}

\begin{enumerate}
    \item Zastosowanie sieci neuronowych do przewidywania struktury molekularnej antybiotyków i porównanie wyników z konwencjonalnymi technikami uczenia maszynowego (np. las losowy, maszyna wektorów nośnych, naiwny klasyfikator Bayesa).

    \begin{itemize}
    \item Konwencjonalne techniki uczenia maszynowego, choć umożliwiają przewidywanie struktury molekularnej antybiotyków, są ograniczone do stosunkowo wąskiego zakresu cech molekularnych przedstawianych za pomocą tzw. wektorów fingerprint (odzwierciedlenie obecności lub braku określonych grup funkcyjnych) i deskryptorów (odzwierciedlenie określonych właściwości fizykochemicznych), które muszą być wcześniej zdefiniowane przez człowieka, posiadającego wiedzę dziedzinową. W przeciwieństwie do technik konwencjonalnych, sieci neuronowe uczą się optymalnych reprezentacji molekularnych dla konkretnego zadania predykcyjnego w sposób automatyczny.
    \item Sieci neuronowe nie tylko przewyższają konwencjonalne techniki uczenia maszynowego w wielu zadaniach związanych z przewidywaniem właściwości fizykochemicznych i biologicznych, lecz także są bardziej odporne na zmiany i lepiej uogólniają się na przestrzenie chemiczne wykraczające poza te, na których model był pierwotnie trenowany.
    \item Sieci neuronowe umożliwiają odkrywanie antybiotyków o odmiennej strukturze w porównaniu do aktualnie stosowanych antybiotyków (np. halicyna i wiele innych).
    \item Odkrycie nowego antybiotyku po 30 latach stagnacji w tym obszarze  (konkretnie halicyny, charakteryzującej się odmienną strukturą molekularną w porównaniu z aktualnie stosowanymi antybiotykami oraz działaniem bakteriobójczym wobec szerokiego spektrum patogenów) zostało dokonane dzięki zastosowaniu głębokich sieci neuronowych (ang. deep neural network) \- istotna wskazówka, że do przewidywania antybiotyków przeciwko innym bakteriom warto zawsze zastosować sieci neuronowe.
    \item Porównanie różnych technik uczenia maszynowego może umożliwić dodatkowe lepsze zrozumienie, które z nich charakteryzują się największą generalizacją, dokładnością i precyzją.
    \end{itemize}

\end{enumerate}