\section{Szablon artykułu naukowego}

\subsection{Tytuł roboczy}
“A Deep Learning Approach to Antibiotic Discovery”

\subsection{Autorzy}
Jonathan M. Stokes, Kevin Yang, Kyle Swanson, Wengong Jin, Andres Cubillos-Ruiz, Nina M. Donghia, Craig R. MacNair, Shawn French, Lindsey A. Carfrae, Zohar Bloom-Ackermann, Victoria M. Tran, Anush Chiappino-Pepe, Ahmed H. Badran, Ian W. Andrews, Emma J. Chory, George M. Church, Eric D. Brown, Tommi S. Jaakkola, Regina Barzilay, James J. Collins

\subsection{Data}
20 lutego 2020

\subsection{Historia i przekaz główny}
Przekaz jest taki, że interdyscyplinarność umożliwiła opracowanie nowego antybiotyku, co przypomnę, ostatni raz miało miejsce ponad 30 lat temu. Dodatkowo 8 z 23 wytypowanych związków będzie jeszcze badanych laboratoryjnie, gdyż wykazały one pożądane właściwości.

\subsection{Tytuł roboczy}
Prawdopodobnie taki jak wersji ostatecznej: “A Deep Learning Approach to Antibiotic Discovery”

\subsection{Lista referencji}
Bibliografia liczyła 70 pozycji. Zarówno dotyczących medycyny, biologii i sieci neuronowych. Najstarsza z pozycji jest z roku 1977 (A protonmotive force drives bacterial flagella) a najnowsze to kilka pozycji z roku 2019, czyli mniej niż rok przed publikacją. Pozwala to przypusczać że autorzy dogłębnie zbadali zarówno najnowsze odkrycia z medycyny (A multiplexable assay for screening antibiotic lethality against drug-tolerant bacteria) jak i zagadnień uczenia maszynowego (A white-box machine learning approach for revealing antibiotic mechanisms of action)

\subsection{Wprowadzenie}
Artykuł wprowadza nas w problem jakim są szczepy bakterii odporne na działanie antybiotyków. Następnie nakreśla pomysł jakim było wykorzystanie baz danych cząsteczek (molekuł) i wyszukanie w nich, z pomocą Uczenia Maszynowego, związków o właściwościach. Do badań laboratoryjnych wytypowano 23 związki z ponad 107 milionów (baza ZINC15).

\subsection{Metody}
Cały opisany proces liczył wiele metod:
\begin{enumerate}
    \item Dobór i przygotowanie danych: Zbiór uczący składał się z 2335 związków: leków zatwierdzonych przez FDA oraz naturalnych produktów, testowanych pod kątem zahamowania wzrostu Escherichia coli. Wytrenowany zaś model testował ponad 107 milionów molekuł z bazy ZINC15.
    \item Reprezentacja molekuł: Cechy 2335 molekuł zostały zwektoryzowane a wiązania międzycząsteczkowe przedstawione w formach grafów skierowanych.
    \item Trening i walidacja modelu: Do treningu użyto sieci konwolucyjnej a konkretnie Directed Message Passing Neural Network (D-MPNN) a skuteczność modelu oceniono głównie za pomocą krzywych ROC-AUC (wartość 0.896)
    \item Predykcja i selekcja: z bazy ponad 107 milionów cząsteczek (molekuł) do testów empirycznych wybrano 23 związki.
    \item Walidacja empiryczna: Przeprowadzono zarówno testy in vitro (czyli wykazano bakteriobójczość na poziomie komórkowym) jak i testy in vivo (czyli wykazano skuteczność na organizmach żywych. W tym wypadku potwierdzono skuteczność Hialicyny na myszach)
\end{enumerate}

\subsection{Wyniki}
Wynikiem całego procesu opisanego w artykule jest niewątpliwie opracowanie Hialicyny. Dodatkową wartością są: 23 związki przeznaczone do dalszych badań laboratoryjnych, wytrenowany już model Uczenia Maszynowego oraz empiryczny dowód na to, że warto stosować algorytmy Uczenia Maszynowego w dziedzinach, gdzie pozornie nie widzi się już pola do odkryć.

Sama hialicyna prócz skuteczności wobec E.Coli okazała się jeszcze wysoce skuteczna wobec A. baumannii, M. tuberculosis oraz P. aeruginos.

\subsection{Dyskusja}
Wstępem do części poświęconej dyskusji było nakreślenie problemu jakim była odporność niektórych bakterii na wcześniej znane antybiotyki. Dyskusja uzasadniała dobór danych, zastosowane metody nauki czy wytypowane predykcje.

Dalsza dyskusja podkreślała niedoskonałości sieci neuronowych jak i potencjalne niskie zróżnicowanie danych. Kolejnym poruszonym problemem były już same predykcje i czy faktycznie będą one miały właściwości bakteriobójcze.

W ostatnim akapicie dyskusja dawała potwierdzenie założonym wnioskom m.in. poprzez badania laboratoryjne z dziedziny biomedycznej.

\subsection{Rysunki/tabele}
Każdy z etapów przedstawiał wyniki czyniąc je czytelnymi:
\begin{itemize}
    \item Schemat blokowy całego postępowania udostępniony był w formie graficznej (rysunek 1.)
    \item Wyniki walidacji treningu sieci neuronowej i najistotniejszych badań laboratoryjnych opatrzone dziewięcioma wykresami m.in. krzywe ROC-AUC, wartości OD dla 99 najlepszych predykcji, spadek rozwoju bakterii E.coli zależny od stężenia Hialicyny.
    \item Szczegółowe wyniki badania opatrzone kilkudziesięcioma wykresami m.in. spadki liczebności pozostałych opisanych bakterii od stężenia Hialicyny.
    \item Tabelaryczne wykazy testowanych bakterii
    \item Tabelaryczne wykazy innych związków niż Hialicyna przeznaczonych do dalszych testów
    \item Tabelaryczny wykaz testowanych algorytmów i modeli (m.in. Chemprop, RDKit, BWA, edgeR)
\end{itemize}

\subsection{Wnioski}
Główne wnioski są takie:
\begin{itemize}
    \item Zastosowanie Uczenia Maszynowego umożliwiło opracowanie nowego antybiotyku (30 lat po ostatnim opracowanym).
    \item Nawet wielkie zbiory danych nie muszą stanowić problemu, jeśli wytrenowany model okaże się skuteczny.
    \item Pomimo koncentracji badań na zwalczaniu bakterii E. Coli Hialicyna okazała się również wysoce skuteczna wobec A. baumannii, M. tuberculosis oraz P. aeruginosa (prócz tego kilka innych również zostało wymienionych)
\end{itemize}