\chapter{Warsztaty 3}
\section{Wprowadzenie i problematyka}
Prowadzenie badań ukierunkowanych na odkrycie nowych struktur antybiotyków jest ważne ze względu na zdolność bakterii do przekazywania genu odporności na antybiotyki. Prowadzi to do lekooporności i obniża skuteczność funkcjonujących na rynku leków. Rozszerzanie wiedzy dotyczącej możliwych rozwiązań tego problemu nie leży w interesie firm farmaceutycznych, dlatego ważne jest, aby tematyka ta była podejmowana przez instytuty badawcze. Wcześniejsze próby skupiały się na analizie molekularnych struktur cząsteczek i przebadanych już antybiotyków. Było to jednak niewystarczające, ponieważ wśród podobnych struktur przez wiele lat nie znaleziono nowej cząsteczki spełniającej wymagania. Dlatego konieczne stało się poszukiwanie wśród niekonwencjonalnych struktur molekularnych. \\
Z uwagi na ogromny rozmiar przestrzeni możliwych rozwiązań (ponad 107 milionów cząsteczek), aby odkryć nowe struktury cząsteczek o pożądanych właściwościach konieczne jest zastosowanie metod uczenia maszynowego. Po nauczeniu modelu z uwzględnieniem kluczowych parametrów przeszukiwane są bazy struktur cząsteczek chemicznych pod kątem występowania tych cech. Takie podejście pozwoliło na odkrycie hialicyny – nowego antybiotyku i stanowiło przełom w badaniach nad nowymi strukturami molekularnymi antybiotyków.


\section{Rodzaje publikacji naukowych}
Wybrane oryginalne artykuły naukowe (research)
    \begin{itemize}
        \item “A Deep Learning Approach to Antibiotic Discovery” [1] 
        \item “Antibiotic discovery with artificial intelligence for the treatment of Acinetobacter baumannii infections” [2] 
    \end{itemize}
Wybrane artykuły przeglądowe (review)
    \begin{itemize}
        \item “Artificial Intelligence (AI) Applications in Drug Discovery and Drug Delivery: Revolutionizing Personalized Medicine” [3] 
        \item “Artificial intelligence to deep learning: machine intelligence approach for drug discovery”  [4] 
    \end{itemize}
Publikacje naukowe mogą mieć charakter oryginalny lub przeglądowy. W pierwszym przypadku autorzy przedstawiają swoje oryginalne dokonania w danej dziedzinie wraz ze szczegółowym opisem wykorzystanej metodyki oraz uzyskanych wyników. Każdy etap badania jest dokładnie udokumentowany co pozwala innym badaczom na odtworzenie eksperymentu w takich samych warunkach. Struktura artykułów naukowych jest zgodna ze strukturą IMRAD: wstęp (introduction), metody (methods), wyniki i dyskusja (results and discussion). Artykuły przeglądowe skupiają się na ogólnym kierunku badań w wybranej tematyce. Autorzy korzystając z oryginalnych artykułów naukowych porównują wyniki badań i analizują je. Opisują kierunek badań, dokonania w wybranej dziedzinie oraz stan aktualnej wiedzy. Mają bardziej holistyczny charakter a co za tym idzie metodyka jest omówiona w sposób skrótowy. Artykuły przeglądowe to przekrój przez najważniejsze odkrycia w dziedzinie. Stanowią zatem dobry początek zaznajamiania się z nową tematyką np. w ramach przeglądu literaturowego do pracy naukowej. \\

W przypadku wybranych artykułów tematyka skupia się na wykorzystaniu metod uczenia maszynowego do odkrywania nowych struktur molekularnych antybiotyków dla bakterii uznanych za stanowiące największe zagrożenie dla zdrowia i życia człowieka. Artykuły przeglądowe opisują w jaki sposób na przestrzeni lat różne metody uczenia maszynowego od prostych algorytmów, przez deep learning po sztuczną inteligencję były wykorzystywane do znajdywania nowych substancji leczniczych. Wskazują one w jakich przypadkach i z jakim skutkiem każda z metod znajdowała zastosowanie. Stanowią wprowadzenie do uczenia maszynowego i są bazą dla bardziej dokładnego researchu, zapoznają z tematyką i praktycznym wykorzystaniem technik. Oryginalne artykuły naukowe zawierają szczegółowe informacje dotyczące warunków i wyników eksperymentu co jest pomocne w prowadzeniu własnych badań w wybranej dziedzinie.
\section{Wybór literatury}
    \begin{itemize}
        \item Zgodność z tematyką badań - publikacja powinna odnosić się do postawionego problemu badawczego tj. przewidywania struktury molekularnej antybiotyków za pomocą metod uczenia maszynowego. Zawężenie dziedziny poszukiwań pozwala na szybsze zdobycie wiedzy w tym zakresie oraz znalezienie artykułów najlepiej odpowiadającym potrzebom badacza. 
        \item Sprawdzone i wiarygodne źródło - publikacja powinna zostać opublikowana w uznanym czasopiśmie naukowym lub wydawnictwie np. Elsevier, Springer, IEE. Takie źródła publikują pozycje rzetelnie zrecenzowane i skontrolowane pod względem adekwatnej metodologii, oryginalności i wiarygodności wyników. Treści można szukać poprzez bazy i biblioteki oferujące dostęp do artykułów naukowych takie jak ScienceDirect. Dodatkowo dorobek naukowy autorów publikacji można sprawdzić za pomocą systemu ORICID. 
        \item Reprezentuje aktualny stan wiedzy - data wydania publikacji jest ważna ze względu na postęp techniczny w dziedzinach takich jak informatyka i biomedycyna. Wiedza szybko ulega dezaktualizacji, nowe metody są wypracowywane, które lepiej odpowiadają aktualnym potrzebom badaczy i pozwalają uzyskać lepsze wyniki. W tym przypadku są to nowoczesne techniki uczenia maszynowego opierające się o uczenie głębokie i metody sztucznej inteligencji do przeszukiwania licznych baz danych ze strukturami molekularnymi wybierając te predysponujące to hamowania rozwoju bakterii.
        \item Wartość naukowa i Impact Factor - wskaźnik liczby cytowań pozwala na ocenę wpływu czasopisma i publikacji na wybraną dziedzinę nauki. Wybranie publikacji o wysokim współczynniku wskazuje na to, że publikacja została dobrze przyjęta w środowisku naukowym, jest często cytowana i wywarła istotny wpływ na dziedzinę. Dla wybranych artykułów IF wynosił 3.8, 4.8, 45.
        \item Znaczenie i przełomowość badania - warto wybierać publikacje, które wniosły istotny wkład w rozwój wybranej tematyki, zapoczątkowały nowy kierunek badań lub opisują przełomowe odkrycie (np. Odkrycie hialicyny). Dzięki temu uwzględnia się pozycje kluczowe i znaczące dla danej dziedziny.
    \end{itemize}

\section{Wkład w dziedzinę}
Wkład oryginalny dla każdej z wybranych publikacji
    \begin{itemize}
        \item “A Deep Learning Approach to Antibiotic Discovery” [1] 
Badacze wykorzystując metody uczenia głębokiego odkryli nowy antybiotyk - hialicynę o odmiennej strukturze molekularnej od dotychczas dostępnych i przebadanych antybiotyków. Badanie to zapoczątkowało nowy kierunek badań - wykorzystanie metod uczenia maszynowego do znajdywania cząsteczek o bakteriobójczych predyspozycjach. Pozwala to na skrócenie czasu badań laboratoryjnych oraz poszerzenie obszaru poszukiwań o nowe, dotychczas niebadane cząsteczki. Artykuł ten był przełomowy ze względu na wykorzystane w badaniu metody oraz ogromny sukces odkrycia nowego antybiotyku nakierowanego na bakterie Escherichia coli, co przyczynia się do walki z postępującą lekoopornością bakterii na powszechnie dostępne antybiotyki. Został uznany i pozytywnie przyjęty w środowisku badaczy i jest szeroko cytowany. 
        \item “Antibiotic discovery with artificial intelligence for the treatment of Acinetobacter baumannii infections” [2] 
Artykuł ten opisuje wykorzystanie metod sztucznej inteligencji do odkrycia nowego antybiotyku na bakterie Acinetobacter baumannii - groźny lekooporny szczep występujący w szpitalach wywołujący szpitalne zapalenie płuc. Stosując analizę QSAR (Quantative Structure-Activity Relashionship) przeanalizowali ponad 11 tysięcy związków chemicznych pochodzenia naturalnego w celu określenia ich biologicznej aktywności względem bakterii. Badacze skupili się na znalezieniu odpowiedniego związku do wytworzenia stabilnej membrany, która hamowałaby zdolność bakterii do namnażania, adhezji oraz powodowałaby śmierć komórki. Udało się wytypować związek o podanych właściwościach, który zapowiada się obiecująco w badaniach in vitro. Odkrycie to stanowi krok w kierunku nowego leku na jeden z najgroźniejszych szczepów bakterii. 
        \item “Artificial Intelligence (AI) Applications in Drug Discovery and Drug Delivery: Revolutionizing Personalized Medicine” [3] 
Autorzy artykułu analizują wykorzystanie sztucznej inteligencji w przemyśle farmaceutycznym. Zwracają uwagę na jej wpływ w przyspieszenie badań i odkryć nowych leków, potencjał na tworzenie personalizowanych produktów leczniczych i monitorowanie ich bezpieczeństwa w procesach klinicznych. Artykuł stanowi przegląd dotychczasowego wykorzystania sztucznej inteligencji, które zrewolucjonizowało dotychczasowy przemysł farmaceutyczny. W kontekście wybranego tematu, zastosowania skupiają się wokół algorytmów pozwalających na szybsze przeszukanie przestrzeni rozwiązań pod kątem struktury molekularnej, fizykochemicznych właściwości cząsteczki oraz oddziaływań międzycząsteczkowych. Badacze zauważają możliwość zastosowania generatywnych sieci współzawodniczących (GAN) do odkrywania nowych struktur cząsteczek leków o zwiększonej selektywności i wzmocnionym działaniu. Dodatkowo rozwiązania bazujące na AI takie jak Atomwise czy BenevolentAI pozwalają na wybranie cząsteczek o największym prawdopodobieństwie sukcesu w badaniach klinicznych co znacząco skraca czas potrzebny do wprowadzenia nowego produktu na rynek oraz pozwala zaoszczędzić koszty produkcji, życie zwierząt oraz zasoby firmy. Badacze zwracają uwagę na zagrożenia wynikające z wykorzystywania sztucznej inteligencji takie jak kwestie etyki i prywatności oraz konstruowania i rozumienia modeli. 
        \item “Artificial intelligence to deep learning: machine intelligence approach for drug discovery” [4] 
Artykuł ten stanowi wprowadzenie do metod uczenia głębokiego i sztucznej inteligencji wykorzystywanych do projektowania i odkrywania nowych substancji leczniczych. Opisywane są techniki stosowane w konkretnych etapach procesu wraz z konkretnymi publicznymi zasobami. Badacze wskazują tutaj na konkretne implementacje algorytmów wraz z dostępnymi narzędziami i repozytoriami. Stanowi to bardzo dobry punkt startowy dla własnych badań nad strukturą molekularną leków. 
    \end{itemize}